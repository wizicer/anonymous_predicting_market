% \documentclass[11pt,twoside,a4paper]{article}
% \usepackage{times}

% \usepackage{xeCJK}

% \setmainfont{Times New Roman}

% \setCJKmainfont{Songti SC}
\documentclass[10pt]{ctexart}
% \usepackage[UTF-8]{ctex}
\usepackage{amsmath}
\usepackage{amsthm} % 根据 amsthm 的手册, amsthm 的加载要在 amsmath 之后
\usepackage{amssymb}  %为了能使用\mathbb{H} 
\usepackage{booktabs}
\usepackage{multirow}
\usepackage{tabularx}
\usepackage{xcolor}
\usepackage[colorlinks,linkcolor=blue]{hyperref} % 使用超链接
\usepackage{pdfpages}
\usepackage{geometry}
\geometry{a4paper,scale=0.8}
\usepackage{graphicx} %插入图片的宏包
\usepackage{float} %设置图片浮动位置的宏包
\usepackage{subfigure} %插入多图时用子图显示的宏包
\usepackage{graphicx}
\usepackage{enumerate}
% \usepackage{natbib}
% \newcommand{\upcite}[1][{\setcitestyle{square,super}}\cite{#1}]

\usepackage{listings}

\lstset{
 columns=fixed,       
 numbers=left,                                        % 在左侧显示行号
 numberstyle=\tiny\color{gray},                       % 设定行号格式
 frame=none,                                          % 不显示背景边框
 backgroundcolor=\color[RGB]{245,245,244},            % 设定背景颜色
 keywordstyle=\color[RGB]{40,40,255},                 % 设定关键字颜色
 numberstyle=\footnotesize\color{darkgray},           
 commentstyle=\it\color[RGB]{0,96,96},                % 设置代码注释的格式
 stringstyle=\rmfamily\slshape\color[RGB]{128,0,0},   % 设置字符串格式
 showstringspaces=false,                              % 不显示字符串中的空格
%language=c++,                                        % 设置语言
}

\newtheorem{definition}{定义}
\newtheorem{lemma}{引理}
\newtheorem{theorem}{定理}
\newtheorem{example}{例}


\title{隐私预测市场}
\author{Jade Xie}
\date{\today}
\begin{document}
\maketitle
% \tableofcontents

\section{协议描述}

\subsection{阈值加密准备}

阈值加密委员会:
\begin{itemize}
    \item $n$ 个成员的私钥:$sk_1, sk_2, \ldots, sk_n$ 
    \item 公钥:$pk$
    \item 阈值: $t$
\end{itemize}
对于一个用 $pk$ 进行加密的秘文 $C$,当至少有 $n$ 个成员中的 $t$ 个成员的私钥进行解密时,都能解密成功。

\subsection{下注阶段}
对于一个用户,其地址为 $address_i$ ,当其对某一个预测方向 $side_i$ 下注 $amount_i$ 金额时,链下计算:
\begin{itemize}
    \item $ct_i \leftarrow Enc(pk, side_i \| address_i)$: 阈值加密的下注方向的密文,其中 $side_i$ 是下注的方向,$address_i$ 为用户地址。
    \item $comm_i = Poseidon(side_i \| salt \| amount_i \| address_i)$: 计算承诺,其中 $salt$ 为全局公开的随机盐。
    \item $(\pi_i, ct_i) \leftarrow Prove(pk, comm_i, amount_i, address_i, salt;side_i)$: ZK Proof 证明下注合法。证明:
    \begin{enumerate}
        \item $side_i \in \{0,1\}$
        \item $amount_i > 0$,以及验证 $amount_i$ 的取值范围合法。
        \item 计算 $ct_i := Enc(pk, side_i \| address_i)$
        \item $comm_i == Poseidon(side_i \| salt \| amount_i \| address_i)$
    \end{enumerate}
\end{itemize}
接着,向链上提交 $(address_i, amount_i, ct_i, comm_i, \pi_i)$ 。

链上执行:
\begin{itemize}
    \item $1 / 0 \leftarrow Verify(\pi_i, address_i, ammount_i, ct_i, comm_i,pk)$ :验证 ZK Proof 是否有效。若输出为 $1$,表明证明有效,则链上记录下注数据\[\{address_i, ammount_i, ct_i, comm_i\}\] ; 否则提示用户下注失败。
\end{itemize}

\subsection{开奖阶段}

在开奖阶段,阈值解密委员会从链上拉取所有的数据,若这一轮中链上总共有 $M$ 个用户的有效数据,则从链上拉取的数据为

\[
\{(address_i, amount_i, ct_i, comm_i)\}_{i = 1}^{M}
\]
在链下,计算
\begin{itemize}
    \item 对于 $i = 1, \ldots, M$, 计算 $(side_i, address_i) \leftarrow Dec(ct_i, sk_1, sk_2, \ldots, sk_t)$: 解密委员会中要至少有 $t$ 个成员才能对阈值加密的密文 $ct_1, \ldots, ct_M$ 进行解密。
    \item $\left(\pi_{batch}, sum_0, sum_1 \right) \leftarrow BatchProve\left(pk, salt, \{(address_i, comm_i, ammount_i)\}_{i = 1}^M;\{side_i\}_{i = 1}^M\right)$:生成批量证明并输出聚合金额,在电路中:
    \begin{enumerate}
        \item 证明 $side_i \in \{0, 1\}$ ($i = 1, \ldots, M$)
        \item 证明 $comm_i == Poseidon(side_i \| salt \| amount_i \| address_i)$ ($i = 1, \ldots, M$)
        \item 计算输出 $sum_0, sum_1$,\[sum_0 = \sum_{i = 1}^M (1 - side_i) \times amount_i , \quad sum_1 = \sum_{i = 1}^M side_i \times amount_i \]
    \end{enumerate}
\end{itemize}
向链上提交数据 $(\pi_{batch}, sum_0, sum_1)$ , 链上计算
\begin{enumerate}
    \item $1 / 0 \leftarrow BatchVerify(\pi_{batch}, sum_0, sum_1, pk, salt, \{(address_i, amount_i, comm_i)\}_{i = 1}^M)$: 验证批量证明是否有效。若输出为 $1$, 则说明批量证明有效,则继续以下流程,否则直接输出失败。
    \item 更新奖金累计池,更新 \[total_0 \leftarrow total_0 +  sum_0 \quad total_1 \leftarrow total_1 +  sum_1\] 
\end{enumerate}
可以分批次得到批量证明,然后逐步更新奖金累计池,直到所有用户的数据都覆盖且不重复。

当从 Oracle 处获得获胜方向 $winning\_side$ 后, 进行结算和分配奖金阶段:
\begin{enumerate}
    \item 计算赔率 $\rho$: \[\rho = \frac{total_0 + total_1}{(1 - winning\_side) \times total_0 + winning\_side \times total_1}\]
    \item 对于每个用户计算奖金,若 $side_i == winning\_side$,则 \[reward_i = amount_i \times \rho, \] 用户调用合约领取收益; 否则 $reward_i = 0$ 。
\end{enumerate}


\end{document}